\subsection{Min\_25 筛}
要求:积性函数 $f(p)$ 是一个关于 $p$ 的项数较少的多项式或可以快速求值;$f(p^{c})$ 可以快速求值。
时间复杂度:$O\left(\frac{n^{\frac{3}{4}}}{\log{n}}\right)$。
部分符号和结论:
\begin{itemize}
\item $F_{\mathrm{prime}}(n) = \sum_{2 \le p \le n} f(p)$
\item $F_{k}(n) = \sum_{i = 2}^{n} [p_{k} \le \lpf(i)] f(i)$,
  答案为 $F_{1}(n) + f(1)$
\item $F_{k}(n) = \sum_{\substack{k \le i \\ p_{i}^{2} \le n}} \sum_{\substack{c \ge 1 \\ p_{i}^{c + 1} \le n}} (f(p_{i}^{c}) F_{i + 1}(n / p_{i}^{c}) +
  f(p_{i}^{c + 1})) + F_{\mathrm{prime}}(n) - F_{\mathrm{prime}}(p_{k - 1})$
\item 若 $f(p) = \sum p^{s_{i}}$,
  设 $g(p) = p^{s}, G_{k}(n) = \sum_{i = 1}^{n} [p_{k} < \lpf(i) \lor \isprime(i)] g(i)$
\item $G_{k}(n) = G_{k - 1}(n) - [p_{k}^{2} \le n] g(p_{k}) (G_{k - 1}(n / p_{k}) - G_{k - 1}(p_{k - 1})), G_{0} = \sum_{i = 2}^{n} g(i)$
\end{itemize}

LOJ 例题:给定 $f(n)$:
$f(n) = \begin{cases} 1 & n = 1 \\ p \oplus c & n = p^{c} \\ f(a)f(b) & n = ab \land a \perp b \end{cases}$,
求 $f(n)$ 的和。
因 $f(p) = p - 1 + 2[p = 2]$,可以按照筛 $\varphi$ 的方法来处理。
\begin{lstlisting}
constexpr int MOD = 1e9 + 7;
constexpr int INV2 = MOD / 2 + 1;

template <typename X, typename Y> void inc(X &x, const Y &y) { x += y, (x >= MOD) && (x -= MOD); }
template <typename X, typename Y> void dec(X &x, const Y &y) { x -= y, (x < 0) && (x += MOD); }
template <typename X, typename Y> int sum(X x, Y y) { return x + y < MOD ? x + y : x + y - MOD; }
template <typename X, typename Y> int sub(X x, Y y) { return x < y ? x + MOD - y : x - y;}

constexpr int MAX_SIZE = 2e5 + 3;
int prime[MAX_SIZE / 10], lpf[MAX_SIZE], spri[MAX_SIZE], prime_cnt;

void sieve(int n) {
  for (int i = 2; i <= n; ++i) {
    if (lpf[i] == 0) {
      prime[lpf[i] = ++prime_cnt] = i;
      spri[prime_cnt] = sum(spri[prime_cnt - 1], i);
    }
    for (int j = 1, x; j <= lpf[i] && (x = i * prime[j]) <= n; ++j) lpf[x] = j;
  }
}

ll g_n, lis[MAX_SIZE];
int G[MAX_SIZE][2], Fprime[MAX_SIZE], cnt;
int lim, le[MAX_SIZE], ge[MAX_SIZE];
#define idx(x) (x <= lim ? le[x] : ge[g_n / x])

void init(ll n) {
  for (ll i = 1, j, x; i <= n; i = n / j + 1) {
    j = n / i, x = j % MOD;
    lis[++cnt] = j, idx(j) = cnt;
    G[cnt][0] = sub(x, 1);
    G[cnt][1] = (x + 2ll) * (x - 1ll) % MOD * INV2 % MOD;
  }
}
void calcFPrime() {
  for (int k = 1; k <= prime_cnt; ++k) {
    const int p = prime[k];
    const ll sqrp = 1ll * p * p;
    for (int i = 1; lis[i] >= sqrp; ++i) {
      const ll x = lis[i] / p;
      const int id = idx(x);
      dec(G[i][0], sub(G[id][0], k - 1));
      dec(G[i][1], 1ll * p * sub(G[id][1], spri[k - 1]) % MOD);
    }
  }
  // f(p) = g_1(p) - g_0(p)
  for (int i = 1; i <= cnt; ++i) Fprime[i] = sub(G[i][1], G[i][0]);
}

int f_p(int p, int c) { return p ^ c; }
int F(int k, ll n) {
  if (n < prime[k] || n <= 1) return 0;
  const int id = idx(n);
  int res = sub(Fprime[id], sub(spri[k - 1], k - 1));
  // F_prime(p_{k - 1}) = spri[k - 1] - (k - 1)
  if (k == 1) res += 2; // 特殊处理 f(2)
  for (int i = k; i <= prime_cnt && 1ll * prime[i] * prime[i] <= n; ++i) {
    ll pw = prime[i], pw2 = pw * pw;
    for (int c = 1; pw2 <= n; ++c, pw = pw2, pw2 *= prime[i])
      inc(res, (1ll * f_p(prime[i], c) * F(i + 1, n / pw) + f_p(prime[i], c + 1)) % MOD);
  }
  return res;
}
\end{lstlisting}

\href{https://zhuanlan.zhihu.com/p/60378354}{新版 min25 筛},复杂度 $O\left(n^{\frac{2}{3}}\right)$。
\begin{lstlisting}
/******************************
f()函数中(31-37行) 填函数在质数幂次处的表达式
pow_sum()函数中(38-43行) 填幂和函数(如果需要更高次的话可以在这里添加)
202-205行按要求填写
f_p[][0/1/2/3/...]分别代表质数个数/质数和/质数平方和/质数三次方和/...根据自己需要添加
例:如果该函数在质数处表达式为f(p) =
p^2+3*p+1,则表明需要质数个数/质数和/质数平方和,即f_p[][0],f_p[][1],f_p[][2]
******************************/

inline ll f(ll p, int e) { // return f(p^e)
  if (p == 1 || e == 0) return 1;
  ll res = mpow(p, e);
  return res * res + 3 * res + 1;
}
ll pow_sum(ll n, int k) { return sum(i^k),i from 1 to n.
  if (k == 0) return n;
  if (k == 1) return n * (n + 1) / 2;
  if (k == 2) return n * (n + 1) * (2 * n + 1) / 6;
}
ll n, f_p[maxn][3]; // F_prime(id(n/i))
int n_2, n_3, n_6;  // sqrt(n), sqrt3(n), sqrt6(n);
ll val_id[maxn];    // give the id, return the id-th number like 'n/i' ,(val_id[1] = 1)
int val_id_num;     // how many numbers like 'n/i'
int val_id_num_3;   // how many numbers like 'n/i' below n/n_3;
int p[200000 + 100];
bool isp[maxn];
int p_sz_2, p_sz_3, p_sz_6; // pi(n_2), pi(n_3), pi(n_6)
void init() {
  n_2 = (int)sqrt(n);
  n_3 = (int)pow(n, 1.0 / 3.0);
  n_6 = (int)pow(n, 1.0 / 6.0);
  val_id_num = 0;
  for (ll i = 1; i <= n;) {
    val_id[++val_id_num] = i;
    if (i < n) i = n / (n / (i + 1));
  }
  memset(isp, 1, sizeof isp);
  isp[1] = 0;
  for (int i = 2; i <= n_2; i++) {
    if (isp[i]) {
      p[++p_sz_2] = i;
      if (i <= n_3) p_sz_3++;
      if (i <= n_6) p_sz_6++;
    }
    for (int j = 1; j <= p_sz_2 && p[j] * i <= n_2; j++) {
      isp[i * p[j]] = 0;
      if (i % p[j] == 0) break;
    }
  }
}
inline int get_id(ll k) { // give a number like 'n/i', return the id of it
  return k > n_2 ? val_id_num - n / k + 1 : k;
}
ll c[maxn];
void add(int x, ll d) { for (; x < maxn; x += x & -x) c[x] += d; }
ll sum(int x) {
  ll ans = 0;
  for (; x; x &= x - 1) ans += c[x];
  return ans;
}

struct node {
  int k_max;
  ll val, f_val;
};
void update_bfs(int k, int type) {
  queue<node> q;
  while (!q.empty()) q.pop();
  int e = 1;
  for (ll i = p[k]; i < n / n_3; i *= p[k], ++e)
    q.emplace(k, i, type == -1 ? f(p[k], e) : mpow(i, type));
  while (!q.empty()) {
    node hd = q.front(); q.pop();
    if ((hd.val != p[hd.k_max] && type >= 0) || type == -1) {
      ll w = n / hd.val;
      w = n / w;
      if (type == -1) {
        add(get_id(w), hd.f_val);
        add(val_id_num + 1, -1ll * hd.f_val);
      } else {
        add(get_id(w), -1ll * hd.f_val);
        add(val_id_num + 1, hd.f_val);
      }
    }
    for (int i = hd.k_max + 1; hd.val * p[i] < n / n_3 && i <= p_sz_2; i++) {
      ll res = p[i];
      for (int e = 1;; e++) {
        if (hd.val * res < n / n_3) {
          q.emplace(i, hd.val * res, type == -1 ? hd.f_val * f(p[i], e) : hd.f_val * mpow(res, type));
        } else break;
        res *= p[i];
      }
    }
  }
}
void get_f_p(ll n, int times) {
  for (int i = 1; i <= val_id_num; i++)
    for (int j = 0; j <= times; j++)
      f_p[i][j] = pow_sum(val_id[i], j) - 1;
  int now;
  for (now = 1; p[now] <= n_6; now++) {
    for (int j = val_id_num; j >= 1; j--) {
      ll w = val_id[j] / p[now];
      if (w < p[now]) break;
      ll val = 1;
      for (int k = 0; k <= times; k++) {
        f_p[j][k] = f_p[j][k] - val * (f_p[get_id(w)][k] - f_p[p[now - 1]][k]);
        val *= p[now];
      }
    }
  }
  int nnow = now, val = 1;
  for (int tt = 0; tt <= times; tt++) {
    now = nnow;
    memset(c, 0, sizeof c);
    add(1, f_p[1][tt]);
    for (int i = 2; val_id[i] < n / n_3; i++) add(i, f_p[i][tt] - f_p[i - 1][tt]);
    for (; p[now] <= n_3; now++) {
      for (int j = val_id_num; j >= 1; j--) {
        ll w = val_id[j] / p[now];
        if (val_id[j] < n / n_3) break;
        if (w < p[now]) break;
        f_p[j][tt] = w < n / n_3
          ? (f_p[j][tt] - (sum(get_id(w)) - sum(p[now - 1])) * mpow(p[now], tt))
          : f_p[j][tt] = f_p[j][tt] - (f_p[get_id(w)][tt] - sum(p[now - 1])) * mpow(p[now], tt);
      }
      update_bfs(now, tt);
    }
    for (int i = 1; i <= val_id_num && val_id[i] < n / n_3; i++) f_p[i][tt] = sum(i);
    for (; now <= p_sz_2; now++) {
      for (int j = val_id_num; j >= 1; j--) {
        ll w = val_id[j] / p[now];
        if (val_id[j] < n / n_3) break;
        if (w < p[now]) break;
        f_p[j][tt] -= (f_p[get_id(w)][tt] - f_p[p[now - 1]][tt]) * mpow(p[now], tt);
      }
    }
  }

  for (int i = 1; i <= val_id_num; i++) {
    // if f(p) = p^2+3p+1,then write: f_p[i][0] = f_p[i][2] + 3*f_p[i][1] + f_p[i][0];
    f_p[i][0] = f_p[i][2] + 3 * f_p[i][1] + f_p[i][0];
  }
}
ll F[2000000 + 100];
void get_f_3(ll n) { // V(F_{pi(n^(1/3))+1},n)
  ll q = p[p_sz_3 + 1];
  for (int now = 1; now <= val_id_num; now++) {
    if (val_id[now] < q) {
      F[now] = 1;
    } else if (val_id[now] < q * q) {
      F[now] = 1 + (f_p[now][0] - f_p[q - 1][0]);
    } else {
      F[now] = 1 + (f_p[now][0] - f_p[q - 1][0]);
      for (int pp = p_sz_3 + 1; p[pp] <= (int)(sqrt(val_id[now])) && pp <= p_sz_2; pp++) {
        F[now] += f(p[pp], 2) + (f(p[pp], 1)) * (f_p[get_id(val_id[now] / p[pp])][0] - f_p[get_id(p[pp])][0]);
      }
    }
  }
}
void get_f_6(ll n) { // V(F_{pi(n^(1/6))+1},n)
  memset(c, 0, sizeof c), add(1, F[1]);
  for (int i = 2; val_id[i] < n / n_3; i++) add(i, F[i] - F[i - 1]);
  for (int k = p_sz_3; k > p_sz_6; k--) {
    int now = val_id_num;
    for (; val_id[now] >= n / n_3; now--) {
      int e = 1;
      ll _p = p[k];
      while (val_id[now] / _p) {
        if (val_id[now] / _p >= n / n_3) {
          F[now] += F[get_id(val_id[now] / _p)] * f(p[k], e);
        } else {
          F[now] += sum(get_id(val_id[now] / _p)) * f(p[k], e);
        }
        _p *= p[k], e++;
      }
    }
    if (k == 1) break;
    update_bfs(k, -1); // bfs to update [lpf(i)==P{k-1}]f(i)
  }
  for (int i = 1; i <= val_id_num && val_id[i] < n / n_3; i++) F[i] = sum(i);
}
void get_f(ll n) {
  for (int k = p_sz_6; k >= 1; k--) {
    for (int now = val_id_num; now >= 1; now--) {
      int e = 1;
      ll _p = p[k];
      while (val_id[now] / _p) {
        F[now] += F[get_id(val_id[now] / _p)] * f(p[k], e);
        _p *= p[k], e++;
      }
    }
  }
}
int main() { // n = 1000000000; 1e10:455052511,0.83s/0.58s; 1e12:37607912018 9.224s/5.105s
  cin >> n;
  init();
  get_f_p(n, 2), get_f_3(n);
  get_f_6(n), get_f(n);
  for (int i = 1; i <= val_id_num; i++) cout << val_id[i] << " : " << F[i] << endl;
}
\end{lstlisting}