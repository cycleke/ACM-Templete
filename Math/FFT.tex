\subsection{快速傅里叶变换}
\begin{lstlisting}
const int MAXN = 4 * 1e5 + 3;
const double PI = acos(-1);
complex<double> a[MAXN], b[MAXN];

int n, bit;
int rev[MAXN];

void fft(complex<double> *a, int sign) {
  for (int i = 0; i < n; ++i)
    if (i < rev[i]) swap(a[i], a[rev[i]]);
  for (int j = 1; j < n; j <<= 1) {
    complex<double> wn(cos(2 * PI / (j << 1)), sign * sin(2 * PI / (j << 1)));
    for (int i = 0; i < n; i += (j << 1)) {
      complex<double> w(1, 0), t0, t1;
      for (int k = 0; k < j; ++k, w *= wn) {
        t0 = a[i + k], t1 = w * a[i + j + k];
        a[i + k] = t0 + t1, a[i + j + k] = t0 - t1;
      }
    }
  }
  if (sign == -1) for (int i = 0; i < n; ++i) a[i] /= n;
}

int main() {
  int n, m, x;
  cin >> n >> m;
  for (int i = 0; i <= n; ++i) cin >> x, a[i].real(x);
  for (int i = 0; i <= m; ++i) cin >> x, b[i].real(x);

  for (::n = 1, bit = 0; ::n <= n + m; ++bit) ::n <<= 1;
  rev[0] = 0;
  for (int i = 1; i < ::n; ++i) rev[i] = (rev[i >> 1] >> 1) | ((i & 1) << (bit - 1));
  fft(a, 1), fft(b, 1);
  for (int i = 0; i < ::n; ++i) a[i] *= b[i];
  fft(a, -1);
  for (int i = 0; i < n + m + 1; ++i) cout << int(a[i].real() + .5) << " \n"[i = n + m];
  return 0;
}
\end{lstlisting}
