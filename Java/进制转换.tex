\subsection{进制转换}
\begin{lstlisting}[language=java]
import java.io.*;
import java.util.*;
import java.math.*;

public class Main {
  public static void main(String[] args) {
    InputStream inputStream = System.in;
    OutputStream outputStream = System.out;
    Scanner in = new Scanner(inputStream);
    PrintWriter out = new PrintWriter(outputStream);
    Solver solver = new Solver();
    int testCount = Integer.parseInt(in.next());
    for (int i = 1; i <= testCount; i++)
      solver.solve(i, in, out);
    out.close();
  }

  static class Solver {
    public void solve(int testNumber, Scanner in, PrintWriter out) {
      int a = in.nextInt();
      int b = in.nextInt();
      String num = in.next();

      BigInteger value = BigInteger.ZERO;
      for (int i = 0; i < num.length(); ++i) {
        value = value.multiply(BigInteger.valueOf(a));
        value = BigInteger.valueOf(getValue(num.charAt(i))).add(value);
      }
      out.println(a + " " + num);

      if (value.equals(BigInteger.ZERO)) {
        out.println(b + " 0");
        out.println();
        return;
      }

      out.print(b + " ");

      char[] ans = new char[1000];
      int length = 0;
      while (!value.equals(BigInteger.ZERO)) {
        int digit = value.mod(BigInteger.valueOf(b)).intValue();
        value = value.divide(BigInteger.valueOf(b));
        ans[length] = getChar(digit);
        ++length;
      }

      for (int i = length - 1; i >= 0; --i) {
        out.print(ans[i]);
      }
      out.println("\n");
    }

    private int getValue(char ch) {
      if (ch >= 'A' && ch <= 'Z') {
        return ch - 'A' + 10;
      }
      if (ch >= 'a' && ch <= 'z') {
        return ch - 'a' + 36;
      }
      return ch - '0';
    }

    private char getChar(int x) {
      if (x < 10) {
        return (char) ('0' + x);
      } else if (x < 36) {
        return (char) ('A' + x - 10);
      } else {
        return (char) ('a' + x - 36);
      }
    }
  }
}
\end{lstlisting}
